% Options for packages loaded elsewhere
% Options for packages loaded elsewhere
\PassOptionsToPackage{unicode}{hyperref}
\PassOptionsToPackage{hyphens}{url}
\PassOptionsToPackage{dvipsnames,svgnames,x11names}{xcolor}
%
\documentclass[
  spanish,
  11pt,
  a4paper,
  DIV=11,
  numbers=noendperiod]{scrartcl}
\usepackage{xcolor}
\usepackage[margin=2.5cm]{geometry}
\usepackage{amsmath,amssymb}
\setcounter{secnumdepth}{5}
\usepackage{iftex}
\ifPDFTeX
  \usepackage[T1]{fontenc}
  \usepackage[utf8]{inputenc}
  \usepackage{textcomp} % provide euro and other symbols
\else % if luatex or xetex
  \usepackage{unicode-math} % this also loads fontspec
  \defaultfontfeatures{Scale=MatchLowercase}
  \defaultfontfeatures[\rmfamily]{Ligatures=TeX,Scale=1}
\fi
\usepackage{lmodern}
\ifPDFTeX\else
  % xetex/luatex font selection
  \setmainfont[]{Times New Roman}
\fi
% Use upquote if available, for straight quotes in verbatim environments
\IfFileExists{upquote.sty}{\usepackage{upquote}}{}
\IfFileExists{microtype.sty}{% use microtype if available
  \usepackage[]{microtype}
  \UseMicrotypeSet[protrusion]{basicmath} % disable protrusion for tt fonts
}{}
\makeatletter
\@ifundefined{KOMAClassName}{% if non-KOMA class
  \IfFileExists{parskip.sty}{%
    \usepackage{parskip}
  }{% else
    \setlength{\parindent}{0pt}
    \setlength{\parskip}{6pt plus 2pt minus 1pt}}
}{% if KOMA class
  \KOMAoptions{parskip=half}}
\makeatother
% Make \paragraph and \subparagraph free-standing
\makeatletter
\ifx\paragraph\undefined\else
  \let\oldparagraph\paragraph
  \renewcommand{\paragraph}{
    \@ifstar
      \xxxParagraphStar
      \xxxParagraphNoStar
  }
  \newcommand{\xxxParagraphStar}[1]{\oldparagraph*{#1}\mbox{}}
  \newcommand{\xxxParagraphNoStar}[1]{\oldparagraph{#1}\mbox{}}
\fi
\ifx\subparagraph\undefined\else
  \let\oldsubparagraph\subparagraph
  \renewcommand{\subparagraph}{
    \@ifstar
      \xxxSubParagraphStar
      \xxxSubParagraphNoStar
  }
  \newcommand{\xxxSubParagraphStar}[1]{\oldsubparagraph*{#1}\mbox{}}
  \newcommand{\xxxSubParagraphNoStar}[1]{\oldsubparagraph{#1}\mbox{}}
\fi
\makeatother

\usepackage{color}
\usepackage{fancyvrb}
\newcommand{\VerbBar}{|}
\newcommand{\VERB}{\Verb[commandchars=\\\{\}]}
\DefineVerbatimEnvironment{Highlighting}{Verbatim}{commandchars=\\\{\}}
% Add ',fontsize=\small' for more characters per line
\usepackage{framed}
\definecolor{shadecolor}{RGB}{241,243,245}
\newenvironment{Shaded}{\begin{snugshade}}{\end{snugshade}}
\newcommand{\AlertTok}[1]{\textcolor[rgb]{0.68,0.00,0.00}{#1}}
\newcommand{\AnnotationTok}[1]{\textcolor[rgb]{0.37,0.37,0.37}{#1}}
\newcommand{\AttributeTok}[1]{\textcolor[rgb]{0.40,0.45,0.13}{#1}}
\newcommand{\BaseNTok}[1]{\textcolor[rgb]{0.68,0.00,0.00}{#1}}
\newcommand{\BuiltInTok}[1]{\textcolor[rgb]{0.00,0.23,0.31}{#1}}
\newcommand{\CharTok}[1]{\textcolor[rgb]{0.13,0.47,0.30}{#1}}
\newcommand{\CommentTok}[1]{\textcolor[rgb]{0.37,0.37,0.37}{#1}}
\newcommand{\CommentVarTok}[1]{\textcolor[rgb]{0.37,0.37,0.37}{\textit{#1}}}
\newcommand{\ConstantTok}[1]{\textcolor[rgb]{0.56,0.35,0.01}{#1}}
\newcommand{\ControlFlowTok}[1]{\textcolor[rgb]{0.00,0.23,0.31}{\textbf{#1}}}
\newcommand{\DataTypeTok}[1]{\textcolor[rgb]{0.68,0.00,0.00}{#1}}
\newcommand{\DecValTok}[1]{\textcolor[rgb]{0.68,0.00,0.00}{#1}}
\newcommand{\DocumentationTok}[1]{\textcolor[rgb]{0.37,0.37,0.37}{\textit{#1}}}
\newcommand{\ErrorTok}[1]{\textcolor[rgb]{0.68,0.00,0.00}{#1}}
\newcommand{\ExtensionTok}[1]{\textcolor[rgb]{0.00,0.23,0.31}{#1}}
\newcommand{\FloatTok}[1]{\textcolor[rgb]{0.68,0.00,0.00}{#1}}
\newcommand{\FunctionTok}[1]{\textcolor[rgb]{0.28,0.35,0.67}{#1}}
\newcommand{\ImportTok}[1]{\textcolor[rgb]{0.00,0.46,0.62}{#1}}
\newcommand{\InformationTok}[1]{\textcolor[rgb]{0.37,0.37,0.37}{#1}}
\newcommand{\KeywordTok}[1]{\textcolor[rgb]{0.00,0.23,0.31}{\textbf{#1}}}
\newcommand{\NormalTok}[1]{\textcolor[rgb]{0.00,0.23,0.31}{#1}}
\newcommand{\OperatorTok}[1]{\textcolor[rgb]{0.37,0.37,0.37}{#1}}
\newcommand{\OtherTok}[1]{\textcolor[rgb]{0.00,0.23,0.31}{#1}}
\newcommand{\PreprocessorTok}[1]{\textcolor[rgb]{0.68,0.00,0.00}{#1}}
\newcommand{\RegionMarkerTok}[1]{\textcolor[rgb]{0.00,0.23,0.31}{#1}}
\newcommand{\SpecialCharTok}[1]{\textcolor[rgb]{0.37,0.37,0.37}{#1}}
\newcommand{\SpecialStringTok}[1]{\textcolor[rgb]{0.13,0.47,0.30}{#1}}
\newcommand{\StringTok}[1]{\textcolor[rgb]{0.13,0.47,0.30}{#1}}
\newcommand{\VariableTok}[1]{\textcolor[rgb]{0.07,0.07,0.07}{#1}}
\newcommand{\VerbatimStringTok}[1]{\textcolor[rgb]{0.13,0.47,0.30}{#1}}
\newcommand{\WarningTok}[1]{\textcolor[rgb]{0.37,0.37,0.37}{\textit{#1}}}

\usepackage{longtable,booktabs,array}
\usepackage{calc} % for calculating minipage widths
% Correct order of tables after \paragraph or \subparagraph
\usepackage{etoolbox}
\makeatletter
\patchcmd\longtable{\par}{\if@noskipsec\mbox{}\fi\par}{}{}
\makeatother
% Allow footnotes in longtable head/foot
\IfFileExists{footnotehyper.sty}{\usepackage{footnotehyper}}{\usepackage{footnote}}
\makesavenoteenv{longtable}
\usepackage{graphicx}
\makeatletter
\newsavebox\pandoc@box
\newcommand*\pandocbounded[1]{% scales image to fit in text height/width
  \sbox\pandoc@box{#1}%
  \Gscale@div\@tempa{\textheight}{\dimexpr\ht\pandoc@box+\dp\pandoc@box\relax}%
  \Gscale@div\@tempb{\linewidth}{\wd\pandoc@box}%
  \ifdim\@tempb\p@<\@tempa\p@\let\@tempa\@tempb\fi% select the smaller of both
  \ifdim\@tempa\p@<\p@\scalebox{\@tempa}{\usebox\pandoc@box}%
  \else\usebox{\pandoc@box}%
  \fi%
}
% Set default figure placement to htbp
\def\fps@figure{htbp}
\makeatother



\ifLuaTeX
\usepackage[bidi=basic]{babel}
\else
\usepackage[bidi=default]{babel}
\fi
\ifPDFTeX
\else
\babelfont{rm}[]{Times New Roman}
\fi
% get rid of language-specific shorthands (see #6817):
\let\LanguageShortHands\languageshorthands
\def\languageshorthands#1{}


\setlength{\emergencystretch}{3em} % prevent overfull lines

\providecommand{\tightlist}{%
  \setlength{\itemsep}{0pt}\setlength{\parskip}{0pt}}



 


\usepackage[hidelinks]{hyperref}
\KOMAoption{captions}{tableheading}
\makeatletter
\@ifpackageloaded{caption}{}{\usepackage{caption}}
\AtBeginDocument{%
\ifdefined\contentsname
  \renewcommand*\contentsname{Tabla de contenidos}
\else
  \newcommand\contentsname{Tabla de contenidos}
\fi
\ifdefined\listfigurename
  \renewcommand*\listfigurename{Listado de Figuras}
\else
  \newcommand\listfigurename{Listado de Figuras}
\fi
\ifdefined\listtablename
  \renewcommand*\listtablename{Listado de Tablas}
\else
  \newcommand\listtablename{Listado de Tablas}
\fi
\ifdefined\figurename
  \renewcommand*\figurename{Figura}
\else
  \newcommand\figurename{Figura}
\fi
\ifdefined\tablename
  \renewcommand*\tablename{Tabla}
\else
  \newcommand\tablename{Tabla}
\fi
}
\@ifpackageloaded{float}{}{\usepackage{float}}
\floatstyle{ruled}
\@ifundefined{c@chapter}{\newfloat{codelisting}{h}{lop}}{\newfloat{codelisting}{h}{lop}[chapter]}
\floatname{codelisting}{Listado}
\newcommand*\listoflistings{\listof{codelisting}{Listado de Listados}}
\makeatother
\makeatletter
\makeatother
\makeatletter
\@ifpackageloaded{caption}{}{\usepackage{caption}}
\@ifpackageloaded{subcaption}{}{\usepackage{subcaption}}
\makeatother
\usepackage{bookmark}
\IfFileExists{xurl.sty}{\usepackage{xurl}}{} % add URL line breaks if available
\urlstyle{same}
\hypersetup{
  pdftitle={GLM-GLMM},
  pdfauthor={Santos G},
  pdflang={es},
  colorlinks=true,
  linkcolor={blue},
  filecolor={Maroon},
  citecolor={Blue},
  urlcolor={Blue},
  pdfcreator={LaTeX via pandoc}}


\title{GLM-GLMM}
\author{Santos G}
\date{}
\begin{document}
\maketitle

\renewcommand*\contentsname{Tabla de contenidos}
{
\hypersetup{linkcolor=}
\setcounter{tocdepth}{2}
\tableofcontents
}

\section{Contexto de proyecto}\label{contexto-de-proyecto}

En esta sección se introducen los Modelos Lineales Generalizados (GLM) y
Modelos Lineales Mixtos Generalizados (GLMM) aplicados a datos
ecológicos. A diferencia de los modelos lineales simples o múltiples,
los GLM permiten modelar variables de respuesta que no siguen una
distribución normal, como conteos (Poisson o binomial negativa),
proporciones o datos binarios (presencia/ausencia). Los GLMM extienden
aún más este marco al incorporar efectos aleatorios, útiles para
controlar la variabilidad entre grupos o unidades experimentales (por
ejemplo, sitios de muestreo, individuos o años).

\section{Carga de librerías y
dataset}\label{carga-de-libreruxedas-y-dataset}

\begin{Shaded}
\begin{Highlighting}[numbers=left,,]
\CommentTok{\# Librerías necesarias (carga al inicio del documento)}
\FunctionTok{library}\NormalTok{(tidyverse)     }\CommentTok{\# manipulación y ggplot2}
\FunctionTok{library}\NormalTok{(broom)         }\CommentTok{\# tidy(), glance()}
\FunctionTok{library}\NormalTok{(knitr)         }\CommentTok{\# kable()}
\FunctionTok{library}\NormalTok{(lme4)          }\CommentTok{\# glmer()}
\FunctionTok{library}\NormalTok{(broom.mixed)   }\CommentTok{\# tidy() para modelos mixtos}
\FunctionTok{library}\NormalTok{(pROC)          }\CommentTok{\# AUC / ROC (diagnóstico)}
\FunctionTok{library}\NormalTok{(janitor)       }\CommentTok{\# Clean names}
\FunctionTok{library}\NormalTok{(palmerpenguins) }\CommentTok{\# dataset de pingüinos}

\CommentTok{\# Cargar dataset}
\NormalTok{df\_raw }\OtherTok{\textless{}{-}}\NormalTok{ penguins }\SpecialCharTok{\%\textgreater{}\%} \FunctionTok{as\_tibble}\NormalTok{()}
\end{Highlighting}
\end{Shaded}

\section{Preparación de datos (crear variable binaria y
limpiar)}\label{preparaciuxf3n-de-datos-crear-variable-binaria-y-limpiar}

\begin{Shaded}
\begin{Highlighting}[numbers=left,,]
\CommentTok{\# Crear variable binaria: "pico largo" (ejemplo umbral = 45 mm)}
\NormalTok{df\_glm }\OtherTok{\textless{}{-}}\NormalTok{ df }\SpecialCharTok{\%\textgreater{}\%}
\NormalTok{  janitor}\SpecialCharTok{::}\FunctionTok{clean\_names}\NormalTok{() }\SpecialCharTok{\%\textgreater{}\%}
  \FunctionTok{mutate}\NormalTok{(}
    \AttributeTok{long\_bill =} \FunctionTok{case\_when}\NormalTok{(}
      \SpecialCharTok{!}\FunctionTok{is.na}\NormalTok{(bill\_length\_mm) }\SpecialCharTok{\&}\NormalTok{ bill\_length\_mm }\SpecialCharTok{\textgreater{}} \DecValTok{45} \SpecialCharTok{\textasciitilde{}} \DecValTok{1}\NormalTok{,}
      \SpecialCharTok{!}\FunctionTok{is.na}\NormalTok{(bill\_length\_mm) }\SpecialCharTok{\&}\NormalTok{ bill\_length\_mm }\SpecialCharTok{\textless{}=} \DecValTok{45} \SpecialCharTok{\textasciitilde{}} \DecValTok{0}\NormalTok{,}
      \ConstantTok{TRUE} \SpecialCharTok{\textasciitilde{}} \ConstantTok{NA\_real\_}
\NormalTok{    )}
\NormalTok{  ) }\SpecialCharTok{\%\textgreater{}\%}
  \FunctionTok{select}\NormalTok{(species, island, flipper\_length\_mm, bill\_length\_mm, long\_bill) }\SpecialCharTok{\%\textgreater{}\%}
  \FunctionTok{drop\_na}\NormalTok{(flipper\_length\_mm, long\_bill) }\CommentTok{\# quitar filas sin predictor o respuesta}

\CommentTok{\# Comprobar distribución de la variable binaria}
\NormalTok{df\_glm }\SpecialCharTok{\%\textgreater{}\%} 
  \FunctionTok{count}\NormalTok{(long\_bill) }\SpecialCharTok{\%\textgreater{}\%} 
\NormalTok{  knitr}\SpecialCharTok{::}\FunctionTok{kable}\NormalTok{()}
\end{Highlighting}
\end{Shaded}

\begin{longtable}[]{@{}rr@{}}

\caption{\label{tbl-glm-prep}Distribución de la variable respuesta
(long\_bill)}

\tabularnewline

\toprule\noalign{}
long\_bill & n \\
\midrule\noalign{}
\endhead
\bottomrule\noalign{}
\endlastfoot
0 & 177 \\
1 & 165 \\

\end{longtable}

Se observa que 177 pingüinos (≈52\%) tienen pico corto o normal (≤45
mm), mientras que 165 pingüinos (≈48\%) tienen pico largo
(\textgreater45 mm) (ver Tabla~\ref{tbl-glm-prep}). Es decir, la
variable respuesta está balanceada --- hay una proporción similar de
casos ``0'' y ``1''. Esto es excelente, porque un modelo logístico
funciona mejor cuando las clases no están demasiado desbalanceadas (por
ejemplo, 90\% / 10\%).

En términos biológicos, esto indica que dentro del conjunto de datos de
pingüinos del archipiélago Palmer, las longitudes del pico se
distribuyen de forma relativamente equilibrada alrededor del umbral de
45 mm. Por tanto, existen proporciones comparables de individuos con
picos largos y cortos, lo cual permitirá explorar si esta característica
está asociada a diferencias morfológicas (como el largo del aleta) o
taxonómicas (como la especie).

\section{Ajuste del GLM binomial
(logit)}\label{ajuste-del-glm-binomial-logit}

\begin{Shaded}
\begin{Highlighting}[numbers=left,,]
\CommentTok{\# Ajuste GLM binomial: probabilidad de pico largo según flipper\_length y especie}
\NormalTok{modelo\_glm }\OtherTok{\textless{}{-}} \FunctionTok{glm}\NormalTok{(long\_bill }\SpecialCharTok{\textasciitilde{}}\NormalTok{ flipper\_length\_mm }\SpecialCharTok{+}\NormalTok{ species,}
                  \AttributeTok{data =}\NormalTok{ df\_glm,}
                  \AttributeTok{family =} \FunctionTok{binomial}\NormalTok{(}\AttributeTok{link =} \StringTok{"logit"}\NormalTok{))}

\CommentTok{\# Tabla de coeficientes y OR (odds ratios)}
\NormalTok{glm\_coef }\OtherTok{\textless{}{-}}\NormalTok{ broom}\SpecialCharTok{::}\FunctionTok{tidy}\NormalTok{(modelo\_glm, }\AttributeTok{conf.int =} \ConstantTok{TRUE}\NormalTok{) }\SpecialCharTok{\%\textgreater{}\%}
  \FunctionTok{mutate}\NormalTok{(}
    \AttributeTok{estimate =} \FunctionTok{round}\NormalTok{(estimate, }\DecValTok{4}\NormalTok{),}
    \AttributeTok{std.error =} \FunctionTok{round}\NormalTok{(std.error, }\DecValTok{4}\NormalTok{),}
    \AttributeTok{p.value =} \FunctionTok{ifelse}\NormalTok{(p.value }\SpecialCharTok{\textless{}} \FloatTok{0.001}\NormalTok{, }\StringTok{"\textless{} 0.001"}\NormalTok{, }\FunctionTok{round}\NormalTok{(p.value, }\DecValTok{3}\NormalTok{)),}
    \AttributeTok{OR =} \FunctionTok{round}\NormalTok{(}\FunctionTok{exp}\NormalTok{(estimate), }\DecValTok{3}\NormalTok{),}
    \AttributeTok{OR\_lwr =} \FunctionTok{round}\NormalTok{(}\FunctionTok{exp}\NormalTok{(conf.low), }\DecValTok{3}\NormalTok{),}
    \AttributeTok{OR\_upr =} \FunctionTok{round}\NormalTok{(}\FunctionTok{exp}\NormalTok{(conf.high), }\DecValTok{3}\NormalTok{)}
\NormalTok{  )}

\NormalTok{knitr}\SpecialCharTok{::}\FunctionTok{kable}\NormalTok{(glm\_coef)}
\end{Highlighting}
\end{Shaded}

\begin{longtable}[]{@{}
  >{\raggedright\arraybackslash}p{(\linewidth - 18\tabcolsep) * \real{0.1748}}
  >{\raggedleft\arraybackslash}p{(\linewidth - 18\tabcolsep) * \real{0.0874}}
  >{\raggedleft\arraybackslash}p{(\linewidth - 18\tabcolsep) * \real{0.0971}}
  >{\raggedleft\arraybackslash}p{(\linewidth - 18\tabcolsep) * \real{0.0971}}
  >{\raggedright\arraybackslash}p{(\linewidth - 18\tabcolsep) * \real{0.0777}}
  >{\raggedleft\arraybackslash}p{(\linewidth - 18\tabcolsep) * \real{0.1165}}
  >{\raggedleft\arraybackslash}p{(\linewidth - 18\tabcolsep) * \real{0.1068}}
  >{\raggedleft\arraybackslash}p{(\linewidth - 18\tabcolsep) * \real{0.0777}}
  >{\raggedleft\arraybackslash}p{(\linewidth - 18\tabcolsep) * \real{0.0777}}
  >{\raggedleft\arraybackslash}p{(\linewidth - 18\tabcolsep) * \real{0.0874}}@{}}

\caption{\label{tbl-glm-fit}Coeficientes (logit) y Odds Ratios (OR)}

\tabularnewline

\toprule\noalign{}
\begin{minipage}[b]{\linewidth}\raggedright
term
\end{minipage} & \begin{minipage}[b]{\linewidth}\raggedleft
estimate
\end{minipage} & \begin{minipage}[b]{\linewidth}\raggedleft
std.error
\end{minipage} & \begin{minipage}[b]{\linewidth}\raggedleft
statistic
\end{minipage} & \begin{minipage}[b]{\linewidth}\raggedright
p.value
\end{minipage} & \begin{minipage}[b]{\linewidth}\raggedleft
conf.low
\end{minipage} & \begin{minipage}[b]{\linewidth}\raggedleft
conf.high
\end{minipage} & \begin{minipage}[b]{\linewidth}\raggedleft
OR
\end{minipage} & \begin{minipage}[b]{\linewidth}\raggedleft
OR\_lwr
\end{minipage} & \begin{minipage}[b]{\linewidth}\raggedleft
OR\_upr
\end{minipage} \\
\midrule\noalign{}
\endhead
\bottomrule\noalign{}
\endlastfoot
(Intercept) & -33.5120 & 7.1452 & -4.690167 & \textless{} 0.001 &
-48.3808699 & -20.223668 & 0.000 & 0.000 & 0.000 \\
flipper\_length\_mm & 0.1533 & 0.0363 & 4.226267 & \textless{} 0.001 &
0.0854135 & 0.228350 & 1.166 & 1.089 & 1.257 \\
speciesChinstrap & 6.2646 & 0.7741 & 8.092529 & \textless{} 0.001 &
4.9003855 & 7.986774 & 525.631 & 134.342 & 2941.792 \\
speciesGentoo & 1.9542 & 0.9068 & 2.154914 & 0.031 & 0.2595677 &
3.869489 & 7.058 & 1.296 & 47.918 \\

\end{longtable}

\begin{itemize}
\item
  \textbf{Intercepto (β₀ = -33.51, p \textless{} 0.001):}\\
  Representa el log-odds de tener un pico largo para un pingüino
  \emph{Adelie} (grupo de referencia) con una longitud de ala igual a 0
  mm. Aunque no tiene un significado biológico directo, es necesario
  para definir la ecuación logística del modelo.
\item
  \textbf{Longitud del ala (β₁ = 0.153, p \textless{} 0.001):}\\
  El efecto es positivo y altamente significativo, lo que indica que a
  medida que aumenta la longitud del ala, también aumenta la
  probabilidad de tener un pico largo. En términos de odds ratio (OR =
  1.166), por cada aumento de 1 mm en la longitud del ala, las
  probabilidades de tener un pico largo aumentan en aproximadamente 16.6
  \%, manteniendo constante la especie.
\item
  \textbf{Especie Chinstrap (β = 6.26, p \textless{} 0.001):}\\
  En comparación con los pingüinos \emph{Adelie}, los \emph{Chinstrap}
  tienen odds 525 veces mayores de presentar un pico largo. Esto sugiere
  una diferencia morfológica marcada entre especies.
\item
  \textbf{Especie Gentoo (β = 1.95, p = 0.031):}\\
  Los \emph{Gentoo} también tienen mayor probabilidad de presentar picos
  largos respecto a los \emph{Adelie}, con odds 7 veces mayores, aunque
  el efecto es más moderado que en \emph{Chinstrap}.
\end{itemize}

En conjunto, el modelo logístico indica que tanto el tamaño corporal
(longitud del ala) como la especie son predictores significativos de la
probabilidad de poseer un pico largo. Las especies \emph{Chinstrap} y
\emph{Gentoo} muestran mayor probabilidad de presentar picos largos en
comparación con \emph{Adelie}, lo que refleja diferencias alométricas y
adaptativas relacionadas con el hábitat y tipo de dieta. La longitud del
ala, a su vez, actúa como indicador de tamaño corporal general,
reforzando la relación alométrica entre las dimensiones corporales (ver
Tabla~\ref{tbl-glm-fit}).

\section{Resumen del ajuste y
verosimilitud}\label{resumen-del-ajuste-y-verosimilitud}

\begin{Shaded}
\begin{Highlighting}[numbers=left,,]
\CommentTok{\# Resumen del ajuste GLM: desviancias, AIC, logLik}
\NormalTok{glm\_glance }\OtherTok{\textless{}{-}}\NormalTok{ broom}\SpecialCharTok{::}\FunctionTok{glance}\NormalTok{(modelo\_glm) }\SpecialCharTok{\%\textgreater{}\%}
\NormalTok{  tibble}\SpecialCharTok{::}\FunctionTok{as\_tibble}\NormalTok{() }\SpecialCharTok{\%\textgreater{}\%}
  \FunctionTok{select}\NormalTok{(null.deviance, df.null, deviance, df.residual, AIC, logLik) }\SpecialCharTok{\%\textgreater{}\%}
  \FunctionTok{mutate}\NormalTok{(}\FunctionTok{across}\NormalTok{(}\FunctionTok{where}\NormalTok{(is.numeric), round, }\DecValTok{3}\NormalTok{))}

\NormalTok{knitr}\SpecialCharTok{::}\FunctionTok{kable}\NormalTok{(glm\_glance)}
\end{Highlighting}
\end{Shaded}

\begin{longtable}[]{@{}rrrrrr@{}}

\caption{\label{tbl-glm-summary}Resumen del ajuste GLM (desviancias,
AIC, logLik)}

\tabularnewline

\toprule\noalign{}
null.deviance & df.null & deviance & df.residual & AIC & logLik \\
\midrule\noalign{}
\endhead
\bottomrule\noalign{}
\endlastfoot
473.692 & 341 & 166.918 & 338 & 174.918 & -83.459 \\

\end{longtable}

\begin{itemize}
\item
  \textbf{Comparación de desviancias:} La \emph{null deviance} es 473.7
  y la \emph{residual deviance} baja a 166.9. Esa gran reducción (más de
  300 unidades) indica que el modelo con predictores mejora muchísimo
  respecto a un modelo sin ellos. En otras palabras,
  \emph{flipper\_length\_mm} y \emph{species} aportan información muy
  significativa para explicar si un pingüino tiene un pico largo
  (\textgreater45 mm).
\item
  \textbf{AIC (174.918):} Este valor por sí solo no tiene un significado
  absoluto, pero sí se usa para comparar modelos alternativos (por
  ejemplo, con o sin alguna variable). Si probaras un modelo solo con
  \emph{flipper\_length\_mm}, su AIC sería más alto, lo que confirmaría
  que incluir \emph{species} mejora el ajuste.
\item
  \textbf{LogLik (-83.459):} Este valor está relacionado con la
  verosimilitud cuanto más alto (menos negativo), mejor el ajuste. A
  medida que agregas predictores relevantes, el logLik debería aumentar.
\end{itemize}

En conjunto el modelo logra una reducción drástica de la deviance (de
473 a 167), lo cual demuestra que el modelo predice bastante bien la
probabilidad de ``pico largo''. La combinación de
\emph{flipper\_length\_mm} (una medida continua) y \emph{species}
(categoría) explica la mayor parte de la variación observada (ver
Tabla~\ref{tbl-glm-summary}).

\section{Diagnósticos GLM: overdispersion y AUC /
performance}\label{diagnuxf3sticos-glm-overdispersion-y-auc-performance}

\begin{Shaded}
\begin{Highlighting}[numbers=left,,]
\CommentTok{\# Overdispersion:ratio deviance/df.residual}

\NormalTok{dispersion }\OtherTok{\textless{}{-}} \FunctionTok{with}\NormalTok{(}\FunctionTok{summary}\NormalTok{(modelo\_glm), deviance }\SpecialCharTok{/}\NormalTok{ df.residual)}

\CommentTok{\# Predicciones y matriz de confusión simple (umbral 0.5)}
\NormalTok{df\_glm }\OtherTok{\textless{}{-}}\NormalTok{ df\_glm }\SpecialCharTok{\%\textgreater{}\%}
  \FunctionTok{mutate}\NormalTok{(}
    \AttributeTok{fitted\_prob =} \FunctionTok{predict}\NormalTok{(modelo\_glm, }\AttributeTok{type =} \StringTok{"response"}\NormalTok{),}
    \AttributeTok{fitted\_class =} \FunctionTok{ifelse}\NormalTok{(fitted\_prob }\SpecialCharTok{\textgreater{}=} \FloatTok{0.5}\NormalTok{, }\DecValTok{1}\NormalTok{, }\DecValTok{0}\NormalTok{)}
\NormalTok{  )}

\NormalTok{conf\_tab }\OtherTok{\textless{}{-}} \FunctionTok{table}\NormalTok{(}\AttributeTok{Actual =}\NormalTok{ df\_glm}\SpecialCharTok{$}\NormalTok{long\_bill, }\AttributeTok{Predicho =}\NormalTok{ df\_glm}\SpecialCharTok{$}\NormalTok{fitted\_class)}
\NormalTok{accuracy }\OtherTok{\textless{}{-}} \FunctionTok{sum}\NormalTok{(}\FunctionTok{diag}\NormalTok{(conf\_tab)) }\SpecialCharTok{/} \FunctionTok{sum}\NormalTok{(conf\_tab)}

\CommentTok{\# AUC / ROC}
\NormalTok{roc\_obj }\OtherTok{\textless{}{-}}\NormalTok{ pROC}\SpecialCharTok{::}\FunctionTok{roc}\NormalTok{(df\_glm}\SpecialCharTok{$}\NormalTok{long\_bill, df\_glm}\SpecialCharTok{$}\NormalTok{fitted\_prob, }\AttributeTok{quiet =} \ConstantTok{TRUE}\NormalTok{)}
\NormalTok{auc\_val }\OtherTok{\textless{}{-}} \FunctionTok{round}\NormalTok{(pROC}\SpecialCharTok{::}\FunctionTok{auc}\NormalTok{(roc\_obj), }\DecValTok{3}\NormalTok{)}

\CommentTok{\# Tablas resumen}
\NormalTok{diag\_table }\OtherTok{\textless{}{-}} \FunctionTok{tibble}\NormalTok{(}
  \AttributeTok{measure =} \FunctionTok{c}\NormalTok{(}\StringTok{"Dispersion (deviance/df)"}\NormalTok{, }\StringTok{"Accuracy (threshold 0.5)"}\NormalTok{, }\StringTok{"AUC"}\NormalTok{),}
  \AttributeTok{value =} \FunctionTok{c}\NormalTok{(}\FunctionTok{round}\NormalTok{(dispersion, }\DecValTok{3}\NormalTok{), }\FunctionTok{round}\NormalTok{(accuracy, }\DecValTok{3}\NormalTok{), auc\_val)}
\NormalTok{)}

\NormalTok{knitr}\SpecialCharTok{::}\FunctionTok{kable}\NormalTok{(diag\_table)}
\end{Highlighting}
\end{Shaded}

\begin{longtable}[]{@{}lr@{}}

\caption{\label{tbl-glm-diagnostics}Diagnósticos GLM: overdispersion,
accuracy y AUC}

\tabularnewline

\toprule\noalign{}
measure & value \\
\midrule\noalign{}
\endhead
\bottomrule\noalign{}
\endlastfoot
Dispersion (deviance/df) & 0.494 \\
Accuracy (threshold 0.5) & 0.904 \\
AUC & 0.958 \\

\end{longtable}

\begin{Shaded}
\begin{Highlighting}[numbers=left,,]
\NormalTok{knitr}\SpecialCharTok{::}\FunctionTok{kable}\NormalTok{(}\FunctionTok{as.data.frame}\NormalTok{(conf\_tab))}
\end{Highlighting}
\end{Shaded}

\begin{longtable}[]{@{}llr@{}}

\caption{\label{tbl-glm-confusion}Matriz de confusión (GLM, umbral 0.5)}

\tabularnewline

\toprule\noalign{}
Actual & Predicho & Freq \\
\midrule\noalign{}
\endhead
\bottomrule\noalign{}
\endlastfoot
0 & 0 & 148 \\
1 & 0 & 4 \\
0 & 1 & 29 \\
1 & 1 & 161 \\

\end{longtable}

Se muestra los indicadores de desempeño del modelo logístico ajustado.
La razón de dispersión (0.494) indica ausencia de sobre-dispersión, lo
que confirma que el modelo binomial es apropiado. La precisión del
90.4\% y el valor AUC de 0.958 evidencian una alta capacidad predictiva
y discriminatoria del modelo, reflejando que la longitud de las aletas y
la especie son buenos predictores de la probabilidad de presentar un
pico largo (ver Tabla~\ref{tbl-glm-diagnostics}) . Además se observa que
el número de aciertos fue sustancialmente mayor que los errores,
reforzando la validez del modelo (ver Tabla~\ref{tbl-glm-confusion}) .

\section{Predicciones: tabla presentable y ejemplo de
interpretación}\label{predicciones-tabla-presentable-y-ejemplo-de-interpretaciuxf3n}

\begin{Shaded}
\begin{Highlighting}[numbers=left,,]
\CommentTok{\# Predicción para valores representativos de flipper por especie}
\NormalTok{newdata }\OtherTok{\textless{}{-}} \FunctionTok{expand.grid}\NormalTok{(}
  \AttributeTok{flipper\_length\_mm =} \FunctionTok{c}\NormalTok{(}\DecValTok{170}\NormalTok{, }\DecValTok{190}\NormalTok{, }\DecValTok{210}\NormalTok{),}
  \AttributeTok{species =} \FunctionTok{unique}\NormalTok{(df\_glm}\SpecialCharTok{$}\NormalTok{species)}
\NormalTok{) }\SpecialCharTok{\%\textgreater{}\%}
  \FunctionTok{as\_tibble}\NormalTok{()}

\NormalTok{preds }\OtherTok{\textless{}{-}}\NormalTok{ newdata }\SpecialCharTok{\%\textgreater{}\%}
  \FunctionTok{mutate}\NormalTok{(}
    \AttributeTok{fit\_prob =} \FunctionTok{predict}\NormalTok{(modelo\_glm, }\AttributeTok{newdata =}\NormalTok{ newdata, }\AttributeTok{type =} \StringTok{"response"}\NormalTok{),}
    \AttributeTok{lwr =} \ConstantTok{NA\_real\_}\NormalTok{, }\AttributeTok{upr =} \ConstantTok{NA\_real\_}
\NormalTok{  )}

\CommentTok{\# obtener IC por link (logit) y transformar a prob (opcional)}
\NormalTok{pred\_link }\OtherTok{\textless{}{-}} \FunctionTok{predict}\NormalTok{(modelo\_glm, }\AttributeTok{newdata =}\NormalTok{ newdata, }\AttributeTok{se.fit =} \ConstantTok{TRUE}\NormalTok{, }
                     \AttributeTok{type =} \StringTok{"link"}\NormalTok{)}
\NormalTok{crit }\OtherTok{\textless{}{-}} \FunctionTok{qnorm}\NormalTok{(}\FloatTok{0.975}\NormalTok{)}
\NormalTok{preds }\OtherTok{\textless{}{-}}\NormalTok{ preds }\SpecialCharTok{\%\textgreater{}\%}
  \FunctionTok{mutate}\NormalTok{(}
    \AttributeTok{fit =} \FunctionTok{round}\NormalTok{(}\FunctionTok{plogis}\NormalTok{(pred\_link}\SpecialCharTok{$}\NormalTok{fit), }\DecValTok{3}\NormalTok{),}
    \AttributeTok{lwr =} \FunctionTok{round}\NormalTok{(}\FunctionTok{plogis}\NormalTok{(pred\_link}\SpecialCharTok{$}\NormalTok{fit }\SpecialCharTok{{-}}\NormalTok{ crit }\SpecialCharTok{*}\NormalTok{ pred\_link}\SpecialCharTok{$}\NormalTok{se.fit), }\DecValTok{3}\NormalTok{),}
    \AttributeTok{upr =} \FunctionTok{round}\NormalTok{(}\FunctionTok{plogis}\NormalTok{(pred\_link}\SpecialCharTok{$}\NormalTok{fit }\SpecialCharTok{+}\NormalTok{ crit }\SpecialCharTok{*}\NormalTok{ pred\_link}\SpecialCharTok{$}\NormalTok{se.fit), }\DecValTok{3}\NormalTok{)}
\NormalTok{  )}

\NormalTok{knitr}\SpecialCharTok{::}\FunctionTok{kable}\NormalTok{(preds)}
\end{Highlighting}
\end{Shaded}

\begin{longtable}[]{@{}rlrrrr@{}}

\caption{\label{tbl-glm_pred_table}Predicciones (probabilidad de pico
largo) por flipper\_length y especie}

\tabularnewline

\toprule\noalign{}
flipper\_length\_mm & species & fit\_prob & lwr & upr & fit \\
\midrule\noalign{}
\endhead
\bottomrule\noalign{}
\endlastfoot
170 & Adelie & 0.0005840 & 0.000 & 0.005 & 0.001 \\
190 & Adelie & 0.0123920 & 0.004 & 0.042 & 0.012 \\
210 & Adelie & 0.2122326 & 0.056 & 0.552 & 0.212 \\
170 & Gentoo & 0.0041077 & 0.000 & 0.089 & 0.004 \\
190 & Gentoo & 0.0813552 & 0.015 & 0.341 & 0.081 \\
210 & Gentoo & 0.6553492 & 0.522 & 0.768 & 0.655 \\
170 & Chinstrap & 0.2349829 & 0.053 & 0.628 & 0.235 \\
190 & Chinstrap & 0.8683359 & 0.731 & 0.941 & 0.868 \\
210 & Chinstrap & 0.9929876 & 0.965 & 0.999 & 0.993 \\

\end{longtable}

Se presentan las probabilidades predichas de presentar un pico largo en
función de la longitud de las aletas y la especie. Se observa una
relación positiva: al aumentar la longitud de las aletas, la
probabilidad de tener un pico largo también aumenta. Sin embargo, esta
relación varía entre especies. En particular, los pingüinos
\emph{Chinstrap} presentan la mayor probabilidad de tener picos largos,
seguidos por \emph{Gentoo}, mientras que \emph{Adelie} muestra
consistentemente bajas probabilidades. Estos resultados confirman que
tanto el tamaño corporal como la especie influyen significativamente en
la morfología del pico (ver Tabla~\ref{tbl-glm_pred_table}) .

\section{Ajuste GLMM (efecto aleatorio por
isla)}\label{ajuste-glmm-efecto-aleatorio-por-isla}

\begin{Shaded}
\begin{Highlighting}[numbers=left,,]
\CommentTok{\# Ajuste GLMM (binomial) con intercepto aleatorio por isla}
\NormalTok{modelo\_glmm }\OtherTok{\textless{}{-}} \FunctionTok{glmer}\NormalTok{(long\_bill }\SpecialCharTok{\textasciitilde{}}\NormalTok{ flipper\_length\_mm }\SpecialCharTok{+}\NormalTok{ species }\SpecialCharTok{+}\NormalTok{ (}\DecValTok{1} \SpecialCharTok{|}\NormalTok{ island),}
            \AttributeTok{data =}\NormalTok{ df\_glm,}
            \AttributeTok{family =} \FunctionTok{binomial}\NormalTok{(}\AttributeTok{link =} \StringTok{"logit"}\NormalTok{),}
            \AttributeTok{control =} \FunctionTok{glmerControl}\NormalTok{(}\AttributeTok{optimizer =} \StringTok{"bobyqa"}\NormalTok{, }
                        \AttributeTok{optCtrl =} \FunctionTok{list}\NormalTok{(}\AttributeTok{maxfun =} \DecValTok{200000}\NormalTok{)))}

\CommentTok{\# Coeficientes fijados (fixed effects) y efecto aleatorio (varianza)}
\NormalTok{glmm\_tidy }\OtherTok{\textless{}{-}}\NormalTok{ broom.mixed}\SpecialCharTok{::}\FunctionTok{tidy}\NormalTok{(modelo\_glmm, }\AttributeTok{effects =} \StringTok{"fixed"}\NormalTok{, }
          \AttributeTok{conf.int =} \ConstantTok{TRUE}\NormalTok{) }\SpecialCharTok{\%\textgreater{}\%}
  \FunctionTok{mutate}\NormalTok{(}
    \AttributeTok{estimate =} \FunctionTok{round}\NormalTok{(estimate, }\DecValTok{4}\NormalTok{),}
    \AttributeTok{OR =} \FunctionTok{round}\NormalTok{(}\FunctionTok{exp}\NormalTok{(estimate), }\DecValTok{3}\NormalTok{),}
    \AttributeTok{p.value =} \FunctionTok{ifelse}\NormalTok{(p.value }\SpecialCharTok{\textless{}} \FloatTok{0.001}\NormalTok{, }\StringTok{"\textless{} 0.001"}\NormalTok{, }\FunctionTok{round}\NormalTok{(p.value, }\DecValTok{3}\NormalTok{))}
\NormalTok{  )}

\NormalTok{rand\_eff }\OtherTok{\textless{}{-}} \FunctionTok{as.data.frame}\NormalTok{(}\FunctionTok{VarCorr}\NormalTok{(modelo\_glmm))}
\NormalTok{rand\_eff\_tbl }\OtherTok{\textless{}{-}} \FunctionTok{tibble}\NormalTok{(}
  \AttributeTok{term =}\NormalTok{ rand\_eff}\SpecialCharTok{$}\NormalTok{grp,}
  \AttributeTok{variance =} \FunctionTok{round}\NormalTok{(rand\_eff}\SpecialCharTok{$}\NormalTok{vcov, }\DecValTok{4}\NormalTok{)}
\NormalTok{)}

\NormalTok{knitr}\SpecialCharTok{::}\FunctionTok{kable}\NormalTok{(glmm\_tidy)}
\end{Highlighting}
\end{Shaded}

\begin{longtable}[]{@{}
  >{\raggedright\arraybackslash}p{(\linewidth - 16\tabcolsep) * \real{0.0753}}
  >{\raggedright\arraybackslash}p{(\linewidth - 16\tabcolsep) * \real{0.1935}}
  >{\raggedleft\arraybackslash}p{(\linewidth - 16\tabcolsep) * \real{0.0968}}
  >{\raggedleft\arraybackslash}p{(\linewidth - 16\tabcolsep) * \real{0.1075}}
  >{\raggedleft\arraybackslash}p{(\linewidth - 16\tabcolsep) * \real{0.1075}}
  >{\raggedright\arraybackslash}p{(\linewidth - 16\tabcolsep) * \real{0.0860}}
  >{\raggedleft\arraybackslash}p{(\linewidth - 16\tabcolsep) * \real{0.1290}}
  >{\raggedleft\arraybackslash}p{(\linewidth - 16\tabcolsep) * \real{0.1183}}
  >{\raggedleft\arraybackslash}p{(\linewidth - 16\tabcolsep) * \real{0.0860}}@{}}

\caption{\label{tbl-glmm-fit}efectos fijos (coeficientes y OR)}

\tabularnewline

\toprule\noalign{}
\begin{minipage}[b]{\linewidth}\raggedright
effect
\end{minipage} & \begin{minipage}[b]{\linewidth}\raggedright
term
\end{minipage} & \begin{minipage}[b]{\linewidth}\raggedleft
estimate
\end{minipage} & \begin{minipage}[b]{\linewidth}\raggedleft
std.error
\end{minipage} & \begin{minipage}[b]{\linewidth}\raggedleft
statistic
\end{minipage} & \begin{minipage}[b]{\linewidth}\raggedright
p.value
\end{minipage} & \begin{minipage}[b]{\linewidth}\raggedleft
conf.low
\end{minipage} & \begin{minipage}[b]{\linewidth}\raggedleft
conf.high
\end{minipage} & \begin{minipage}[b]{\linewidth}\raggedleft
OR
\end{minipage} \\
\midrule\noalign{}
\endhead
\bottomrule\noalign{}
\endlastfoot
fixed & (Intercept) & -33.5120 & 7.2003934 & -4.654188 & \textless{}
0.001 & -47.6244935 & -19.399470 & 0.000 \\
fixed & flipper\_length\_mm & 0.1533 & 0.0365621 & 4.193833 &
\textless{} 0.001 & 0.0816750 & 0.224996 & 1.166 \\
fixed & speciesChinstrap & 6.2646 & 0.7747393 & 8.086017 & \textless{}
0.001 & 4.7460947 & 7.783017 & 525.631 \\
fixed & speciesGentoo & 1.9542 & 0.9102029 & 2.146946 & 0.032 &
0.1701916 & 3.738122 & 7.058 \\

\end{longtable}

\begin{Shaded}
\begin{Highlighting}[numbers=left,,]
\NormalTok{knitr}\SpecialCharTok{::}\FunctionTok{kable}\NormalTok{(rand\_eff\_tbl)}
\end{Highlighting}
\end{Shaded}

\begin{longtable}[]{@{}lr@{}}

\caption{\label{tbl-glmm-var-fit}Varianza del efecto aleatorio (island)}

\tabularnewline

\toprule\noalign{}
term & variance \\
\midrule\noalign{}
\endhead
\bottomrule\noalign{}
\endlastfoot
island & 0 \\

\end{longtable}

El modelo mixto (GLMM) que incluyó un intercepto aleatorio por isla no
mostró variación atribuible a este factor (varianza ≈ 0) (ver
Tabla~\ref{tbl-glmm-var-fit}), lo que indica que la probabilidad de
presentar un pico largo no difiere significativamente entre islas.

En cuanto a los efectos fijos ver (Tabla~\ref{tbl-glmm-fit}) , todos
resultaron estadísticamente significativos. La longitud de las aletas
tuvo un efecto positivo sobre la probabilidad de presentar un pico
largo: por cada milímetro adicional en la longitud del \emph{flipper},
las probabilidades de tener un pico largo aumentan aproximadamente un
16.6\% (OR = 1.17; p \textless{} 0.001).

Asimismo, se observaron diferencias claras entre especies. Los pingüinos
\emph{Chinstrap} mostraron una probabilidad de presentar pico largo más
de 500 veces superior a la de los \emph{Adelie} (OR = 525.6; p
\textless{} 0.001), mientras que los \emph{Gentoo} presentaron una
probabilidad aproximadamente 7 veces mayor (OR = 7.06; p = 0.032).

En conjunto, estos resultados sugieren que el tamaño de las aletas es un
predictor importante de la presencia de picos largos y que existen
diferencias marcadas entre especies, aunque no entre islas.

\section{Diagnóstico GLMM y comparación con
GLM}\label{diagnuxf3stico-glmm-y-comparaciuxf3n-con-glm}

\begin{Shaded}
\begin{Highlighting}[numbers=left,,]
\CommentTok{\# Extract AIC and logLik for comparison}
\NormalTok{model\_comp }\OtherTok{\textless{}{-}} \FunctionTok{tibble}\NormalTok{(}
  \AttributeTok{model =} \FunctionTok{c}\NormalTok{(}\StringTok{"GLM"}\NormalTok{, }\StringTok{"GLMM"}\NormalTok{),}
  \AttributeTok{AIC =} \FunctionTok{c}\NormalTok{(}\FunctionTok{AIC}\NormalTok{(modelo\_glm), }\FunctionTok{AIC}\NormalTok{(modelo\_glmm)),}
  \AttributeTok{logLik =} \FunctionTok{c}\NormalTok{(}\FunctionTok{logLik}\NormalTok{(modelo\_glm), }\FunctionTok{logLik}\NormalTok{(modelo\_glmm))}
\NormalTok{) }\SpecialCharTok{\%\textgreater{}\%}
  \FunctionTok{mutate}\NormalTok{(}\FunctionTok{across}\NormalTok{(}\FunctionTok{where}\NormalTok{(is.numeric), }\SpecialCharTok{\textasciitilde{}} \FunctionTok{round}\NormalTok{(., }\DecValTok{3}\NormalTok{)))}

\NormalTok{knitr}\SpecialCharTok{::}\FunctionTok{kable}\NormalTok{(model\_comp)}
\end{Highlighting}
\end{Shaded}

\begin{longtable}[]{@{}lrr@{}}

\caption{\label{tbl-glmm-diag-compare}Comparación AIC/logLik entre GLM y
GLMM}

\tabularnewline

\toprule\noalign{}
model & AIC & logLik \\
\midrule\noalign{}
\endhead
\bottomrule\noalign{}
\endlastfoot
GLM & 174.918 & -83.459 \\
GLMM & 176.918 & -83.459 \\

\end{longtable}

\begin{Shaded}
\begin{Highlighting}[numbers=left,,]
\CommentTok{\# ICC aproximado para GLMM (logistic)}
\NormalTok{var\_island }\OtherTok{\textless{}{-}} \FunctionTok{as.data.frame}\NormalTok{(}\FunctionTok{VarCorr}\NormalTok{(modelo\_glmm))}\SpecialCharTok{$}\NormalTok{vcov[}\DecValTok{1}\NormalTok{]}
\NormalTok{icc }\OtherTok{\textless{}{-}} \FunctionTok{round}\NormalTok{(var\_island }\SpecialCharTok{/}\NormalTok{ (var\_island }\SpecialCharTok{+}\NormalTok{ (pi}\SpecialCharTok{\^{}}\DecValTok{2} \SpecialCharTok{/} \DecValTok{3}\NormalTok{)), }\DecValTok{3}\NormalTok{)}

\FunctionTok{tibble}\NormalTok{(}\AttributeTok{stat =} \FunctionTok{c}\NormalTok{(}\StringTok{"ICC (aprox.)"}\NormalTok{), }\AttributeTok{value =} \FunctionTok{c}\NormalTok{(icc)) }\SpecialCharTok{\%\textgreater{}\%}\NormalTok{ knitr}\SpecialCharTok{::}\FunctionTok{kable}\NormalTok{()}
\end{Highlighting}
\end{Shaded}

\begin{longtable}[]{@{}lr@{}}

\caption{\label{tbl-glmm-logistic}Proporción de variación entre islas
(ICC aprox.)}

\tabularnewline

\toprule\noalign{}
stat & value \\
\midrule\noalign{}
\endhead
\bottomrule\noalign{}
\endlastfoot
ICC (aprox.) & 0 \\

\end{longtable}

La comparación entre el modelo lineal generalizado (GLM) y el modelo
mixto (GLMM) mostró valores prácticamente idénticos de log-verosimilitud
(-83.46), con un AIC ligeramente mayor en el GLMM (176.9) respecto al
GLM (174.9) (ver Tabla~\ref{tbl-glmm-diag-compare}).

Este resultado indica que la incorporación del efecto aleatorio por isla
no mejora el ajuste del modelo, sino que introduce un parámetro
adicional sin aportar ganancia en verosimilitud. Por tanto, el modelo
más parsimonioso y adecuado es el GLM simple, que explica la
variabilidad en la probabilidad de presentar pico largo sin necesidad de
considerar diferencias entre islas.

En concordancia, el coeficiente de correlación intraclase (ICC)
calculado para el GLMM fue ≈ 0 (ver Tabla~\ref{tbl-glmm-logistic}), lo
que confirma que no existe una proporción relevante de la variación
atribuible al nivel de agrupamiento ``isla''.

En resumen, tanto el análisis del AIC como el ICC sugieren que el efecto
del contexto espacial (isla) es despreciable, y que la variabilidad en
la presencia de picos largos está determinada principalmente por la
especie y la longitud del flipper.

\section{Conclusiones y recomendaciones prácticas (GLM /
GLMM)}\label{conclusiones-y-recomendaciones-pruxe1cticas-glm-glmm}

El análisis realizado con el conjunto de datos de pingüinos de Palmer
permitió modelar la probabilidad de que un individuo presente un pico
largo (\textgreater{} 45 mm) en función de variables morfométricas y
taxonómicas, aplicando tanto modelos lineales generalizados (GLM) como
modelos mixtos (GLMM).

Los resultados del modelo GLM binomial con enlace logit mostraron que la
longitud del flipper es un predictor significativo: por cada milímetro
adicional, las probabilidades de tener un pico largo aumentan
aproximadamente un 16--17 \% (OR ≈ 1.17). Asimismo, se detectaron
diferencias claras entre especies: \emph{Chinstrap} es la especie con
mayor probabilidad de presentar picos largos, seguida por \emph{Gentoo},
mientras que \emph{Adelie} presenta la menor.

\hfill\break
El modelo demostró un ajuste muy adecuado, sin indicios de
sobre-dispersión (dispersion ≈ 0.49), con una exactitud de clasificación
del 90 \% y una curva ROC de AUC = 0.96, lo que evidencia una excelente
capacidad predictiva y discriminante.

Por otro lado, el modelo mixto (GLMM) que incorporó un intercepto
aleatorio por isla no mejoró sustancialmente el ajuste (AIC ≈ 176.9
frente a 174.9 del GLM) y presentó una varianza del efecto aleatorio
prácticamente nula (0). El ICC ≈ 0 confirmó que las diferencias entre
islas no aportan variabilidad adicional al fenómeno modelado. En
consecuencia, el GLMM no ofrece ventajas sobre el GLM en este contexto,
y el modelo simple resulta más parsimonioso y eficiente.




\end{document}
